% !TeX document-id = {5c05c331-134f-48f1-8db3-fd32c67a647b}
%%%%%%%%%%%%%%%%%%%%%%%%%%%%%%%%%%%%%%%%%
% Journal Article
% LaTeX Template
% Version 2.0 (February 7, 2023)
%
% This template originates from:
% https://www.LaTeXTemplates.com
%
% Author:
% Vel (vel@latextemplates.com)
%
% License:
% CC BY-NC-SA 4.0 (https://creativecommons.org/licenses/by-nc-sa/4.0/)
%
% NOTE: The bibliography needs to be compiled using the biber engine.
%
%%%%%%%%%%%%%%%%%%%%%%%%%%%%%%%%%%%%%%%%%

% Magic comments for TeXStudio
% !TeX program = pdflatex
% !BIB program = biber
% !TeX encoding = utf8
% !TeX spellcheck = en_US

%----------------------------------------------------------------------------------------
%	PACKAGES AND OTHER DOCUMENT CONFIGURATIONS
%----------------------------------------------------------------------------------------

\documentclass[
	a4paper, % Paper size, use either a4paper or letterpaper
	10pt, % Default font size, can also use 11pt or 12pt, although this is not recommended
	unnumberedsections, % Comment to enable section numbering
	twoside, % Two side traditional mode where headers and footers change between odd and even pages, comment this option to make them fixed
]{LTJournalArticle}

% BibLaTeX bibliography file
\bibliography{bibAntoine.bib} 
\bibliography{bibNathan.bib} 
\bibliography{bibSandro.bib}

\runninghead{Real-Time Digital Processing of PPG Signals for Wearable Devices} % A shortened article title to appear in the running head, leave this command empty for no running head

\footertext{} % Text to appear in the footer, leave this command empty for no footer text

\setcounter{page}{1} % The page number of the first page, set this to a higher number if the article is to be part of an issue or larger work

% Main header file
% % % % % % % % % % % % % % % % % % % % % % % % % % % % % % % % % % % % % % % % %
%
% OSTReport -- Additional packages frequently used in reports
%
% % % % % % % % % % % % % % % % % % % % % % % % % % % % % % % % % % % % % % % % %

% Mathematical equations
\usepackage{amsmath}
\usepackage{amssymb}
\usepackage{bm}
\usepackage{MnSymbol}
%\usepackage{breqn}

% Tables
\usepackage{multirow}
\usepackage{tabularx}
\usepackage{booktabs}

% Figures
\usepackage{pdfpages}
\usepackage{epstopdf}
\usepackage{float}
\usepackage{graphicx}
\usepackage{caption}
\usepackage{subcaption}
%\usepackage[outdir=./]{epstopdf}

% Quotation marks
\usepackage{csquotes}
\setquotestyle[quotes]{german}

% Si Units
\usepackage{siunitx}
\sisetup{detect-all,sticky-per,per-mode=symbol}

% Multicolumn documents and sections
\usepackage{multicol}

\PassOptionsToPackage{svgnames,x11names,dvipsnames}{xcolor}
\usepackage[most]{tcolorbox}

%----------------------------------------------------------------------------------------
%	TITLE SECTION
%----------------------------------------------------------------------------------------

\title{Real-Time Digital Processing of PPG Signals for Wearable Devices} % Article title, use manual lines breaks (\\) to beautify the layout

% Authors are listed in a comma-separated list with superscript numbers indicating affiliations
% \thanks{} is used for any text that should be placed in a footnote on the first page, such as the corresponding author's email, journal acceptance dates, a copyright/license notice, keywords, etc
\author{%
	Nathan Hoffman\textsuperscript{1}\thanks{Corresponding author: \href{mailto:nathan.hoffman@students.unibe.ch}{nathan.hoffman@students.unibe.ch} \\ \textbf{Received:} May 2, 2024, \textbf{Published:} \today}, Antoine Biebuyck\textsuperscript{1} \\ 
}

% Affiliations are output in the \date{} command
\date{\footnotesize\textsuperscript{\textbf{1}}ARTORG Center for Biomedical Engineering Research, University of Bern, Bern, Switzerland}

% Full-width abstracta
\renewcommand{\maketitlehookd}{%
	\begin{abstract}
		\noindent In recent times, there has been a desire to monitor vital signs such as heartrate in a continuous and cost-effective manner through use of wearable devices. Photoplethysmography presents itself as a robust technology to accomplish that goal despite its initial drawback of high signal noise. Such noise can be relatively easily removed by the use of well-known digital filtering. Thereafter, a heartbeat detection algorithm can be applied to the filtered signal. While these operations are at first applied to the offline signal, translating the implementation on a microcontroller allows for real-time signal filtering and hearbeat detection. This study presents digital filters that succesfully remove the unwanted noise and an algorithm that successfully counts the peaks of heartbeats. While the filtering was also successfully perfomed in real-time, the heartbeat detection algorithm failed to produce an output when carried out in real-time. Further research will be required to refine the results from this study.
	\end{abstract}
}

%----------------------------------------------------------------------------------------

\begin{document}

\maketitle % Output the title section

%----------------------------------------------------------------------------------------
%	ARTICLE CONTENTS
%----------------------------------------------------------------------------------------

\section{Introduction}

\epigraph{What is real? How do you define 'real'? If you're talking about what you can feel, what you can smell, what you can taste and see, then "real" is simply electrical signals interpreted by your brain.}{-- Morpheus, The Matrix}

%\epigraph{Everything we do, every thought we've ever had, is produced by the human brain.}{-- Dr. Neil deGrasse Tyson}

The human brain's remarkable capacity to navigate and master the intricate complexities of our lives is nothing short of \emph{mind-blowing}\textemdash{}pun intended. Equally thought-provoking, nevertheless, is the question of why it can malfunction.

Dementia is a progressive neurodegenerative disease which results in the loss of cognitive functions such as memory, thought and reasoning skills and negatively interferes with a person's activities of daily life \cite{b1}. The World Health Organization estimates that more than 55 million people world-wide are suffering from dementia, with Alzheimer's disease being the most dominant form \cite{b2}. 

While the trigger and driving force behind the progression of Alzheimer's disease still remain uncertain, the neuropathological hallmarks of the disease are the selective loss of cortical neurons within the hippocampus, the temporal lobe and frontal lobe, the gradual loss of synapses, the deposition of amyloid-\textbeta\ forming senile plaques and the presence of intraneuronal neurofibrillary tangles which contain highly phosphorylated microtubule-associated tau proteins. \cite{b3,b4} 

Current assessment of Alzheimer's disease is based on a qualitative psychometric clinical diagnosis relying on criteria established by the Diagnostic and Statistical Manual of Mental Disorders and the National Institute of Neurological and Communicative Disorders and Stroke/Alzheimer's Disease and Related Disorders Association with a definitive diagnosis only confirmed post-mortem through histological staining of brain tissue \cite{b4,b5}. Considering the wealth of modern scientific advancements, these methods have become rather antiquated and frequently overlook an early diagnosis, crucial for any potential intervention to effectively delay the disease process. Hence the ongoing search for a robust biomarker which can identify, assess and monitor the unfolding of Alzheimer's disease \cite{b4}.

While molecular biomarkers, requiring a blood or cerebrospinal fluid sample, have been proposed, morphological biomarkers, such as hippocampus volume and cortical thinning obtained from magnetic resonance imaging or brain hypometabolism information obtained from positron emission tomography, are also promising \cite{b4,b5}. To leverage and evaluate these biomarkers, the regions of interests must be extracted from the scan, a process called image segmentation. Manual segmentation is the simplest technique but suffers from drawbacks which include the need for trained professionals, being time- and labor-intensive and operator-dependency rendering it ill-suited for routine clinical practice \cite{b6}. Consequently, current research is investigating the use of automated methods.

One such method is atlas-based segmentation. A brain atlas is a map that shows the typical structure of the brain anatomy as it is built from population-based imaging data after registration, warping and overlay to a common reference frame and is based on the assumption that all brains, to a degree of deformation, resemble a prototypical template. Applying this concept of an atlas to segmentation tasks leads to label-based approaches, or probabilistic anatomy maps, in which manual anatomical segmentation labels from a library are mapped to the atlas image \cite{b7}. The registration from atlas space to the coordinate system of a newly-obtained brain image then automatically generates with it the tailored segmentation labels.

Of interest in this research study is the performance comparison between an altas-based and machine-learning based segmentation of several brain regions\textemdash{}the white matter, the gray matter, the hippocampus, the amygdala and the thalamus. Machine learning approaches carry several associated costs such as high computational demands, potentially slow inference times, lack of interpretability and transparency that pose a challenge in clinical adoption, potential to overfit and consequent ineffectiveness on unseen data, and dependency on preprocessing, such as skull stripping, bias field correction and normalization which can introduce additional sources for error \cite{b8,b9} Accordingly, the motivation for the juxtaposition in this study is to evaluate whether a so-called more simplistic atlas-based approach can match or even outperform an esoteric and not without drawbacks machine learning approach, calling into question whether the more sophisticated approach is sufficiently warranted for brain tissue segmentation.




















% !TeX encoding = utf8
% !TeX spellcheck = en_US

\section{Methods}
\subsection{Participant}
One male participant 25 years of age was recruited by the ARTORG Center for Biomedical Engineering Research. The participant was healthy and did not suffer from any known pre-existing medical conditions. This study was conducted at the Institute for Human Centered Engineering HuCE at the Bern University for Applied Sciences in Biel, Bern, Switzerland.

\subsection{Data collection}
The PPG signals were recorded with custom hardware which was designed and developed at the HuCE. The hardware consists of an ST microcontroller development board (Plan-les-Ouates, Geneva, Switzerland) and a AMS OSRAM PPG frontend (Premstätten, Styria, Austria). The PPG signal was recorded on the participant's thumb with a sampling rate $fs$ of 500\,Hz. The data was transmitted to the connected PC and was represented in real time using a  graphical user interface (GUI) programmed in Python. This GUI serves dual purposes: facilitating real-time visualization and enabling manipulation and adjustment of both the frontend and the microcontroller configurations.

\subsection{Data processing}
The PPG signals were processed in MATLAB (The MathWorks Inc., Natick, Massachusetts) Version R2023b with additional use of the signal processing and DSP system toolboxes, where finite impulse response (FIR) and infinite impulse response (IIR) filters were designed using the filter designer and afterwards applied to the signals. 

\subsection{Digital filtering}
In order to assess different filters, a IIR Chebyshev Type II and an FIR equiripple filter were chosen. The selected filters have similar pass-band and stop-band characteristics, which facilitates the comparison between the two. The stop-band characteristics can be seen in Figure \ref{fig:cheby} and \ref{fig:equiripple}. The Chebyshev Type II filter has a monotonic pass-band and an equiripple stop-band. Furthermore, it is always normalized to unity at DC and the magnitude response squared $|H(\Omega)|^2$ of the Chebyshev Type II can be computed with \cite{orfanidis_introduction_1996}
\begin{equation}
	|H(\Omega)|^2 = \frac{C_N^2\big(\frac{\Omega_{stop}}{\Omega}\big)}{C_N^2\big(\frac{\Omega_{stop}}{\Omega}\big) + \epsilon_{stop}^2},
\end{equation}
where $C_N(x)$ is the Chebyshev polynomial of degree N, which is defined by
\begin{equation}
	C_N(x) =
	\begin{cases}
		 \cos(N\cos^{-1}(x))	    	& \text{if } |x| \le 1 \\
		\cosh(N\cosh^{-1}(x)),		& \text{if } |x| > 1.
	\end{cases}
\end{equation} 

\begin{table}% Half width table
	\caption{Used filters applied to the signal.}
	\centering % Horizontally center the table
	\begin{tabularx}{\linewidth}{l|l|X|} % Manually specify column alignments with L{}, R{} or C{} and widths as a fixed amount, usually as a proportion of \linewidth
		\toprule
		Filter & Filter type & Order\\
		\midrule
		Chebyshev Type II & IIR Lowpass & 10 \\
		Equiripple  & FIR Lowpass & 211 \\
		\bottomrule
	\end{tabularx}
	\label{tab:packages}
\end{table} 

The principal limitation of designing FIR filters using the window method is the inability to modulate the approximation error across varying frequency ranges. Therefore, it is frequently advantageous to utilize the minimax strategy, which minimizes the peak error, or to apply an error criterion incorporating frequency weighting in filter design. This methodology produces the most optimal filter achievable for a specified set of requirements. So that a FIR filter with minimum effort can be designed the FIR Equiripple method is used. \cite{noauthor_introduction_nodate} 

In order to design an equiripple filter from a set of frequency domain specifications, the weighted error model is used, where $\epsilon(\hat{\omega})$ is 
\begin{equation}
	\epsilon(\hat{\omega}) = W(\hat{\omega}) |H_d(e^{j\hat{\omega}}) - H(e^{j\hat{\omega}})|,
\end{equation}
where the weighted error is defined in terms of a non-negative error weight $W(\hat{\omega}) \ge 0$ and the difference between the \emph{desired} $H_d(e^{j\hat{\omega}})$ and \emph{realized} $ H(e^{j\hat{\omega}})$ filter's base-band frequency response. 
Equiripple FIR filters satisfy the \emph{minimax error criterion}, which asserts that the optimal solution is achieved when $\delta = \min(\max(|\epsilon(\hat{\omega})|))$, with $\hat{\omega} \in [0, \hat{\omega}_s/2]$, where $\delta$ is called the \emph{minimax error} or \emph{external error}.  \cite{williams_electronic_2006}

The band-pass and band-stop frequencies were set to 8\,Hz and 14\,Hz, respectively with a stop-band attenuation of 80\,dB. Both filters were implemented as fixed point arithmetic on the microcontroller, which then applied the filters during the recording of the signal.

\begin{figure}[H]
	\centering
	\includegraphics[width=\linewidth, trim={2.5cm, 1cm, 4cm, 1cm}, clip]{Figures/chebyII.png}
	\caption{Frequency response of IIR Chebyshev Type II.}
	\label{fig:cheby}
\end{figure}

\begin{figure}[H]
	\centering
	\includegraphics[width=\linewidth, trim={2.5cm, 1cm, 4cm, 1cm}, clip]{Figures/equiripple.png}
	\caption{Frequency response of FIR Equiripple.}
	\label{fig:equiripple}
\end{figure}


\subsection{Filter performance evaluation}
So that both filters could be compared objectively their power spectral denstity was computed. The discrete-time Fourier transform of a random signal's autocorrelation function $R_{xx}(k)$ is defined as its power spectrum $S_{xx}(\omega)$. The frequency content of the random signal $x(n)$ is represented by the power spectrum 
\begin{equation}
	S_{xx}(\omega) = \sum_{-\infty}^{\infty} R_{xx}(k)\cdot e^{-j\omega k},
\end{equation}
where $\omega = 2\pi f/f_s$ is the digital or normalized frequency in radians per sample.

Furthermore, the quantity $S_{xx}(f)/f_s$ represents the power per unit frequency interval, which describes the \emph{power spectrum} or \emph{power spectrum density}. It shows how the signal's power is spread out among the individual frequencies. \cite{orfanidis_introduction_1996}

The mentioned filters should have a large amount of their power in the pass-band region and much of the power in the stop-band region should be absent, so that the high frequency noise is attenuated. This observation is corroborated by an objective visual inspection of the power spectra of both filters. The power spectral density plots of both filters can be seen in Figure \ref{fig:powerspectra}.

\begin{figure}
	\centering
	\includegraphics[width=\linewidth, trim={2.5cm, 1cm, 2cm, 1cm}, clip]{Figures/powerspectra.png}
	\caption{Power spectrum of the IIR Chebyshev Type II and FIR Equiripple filter compared to the unfiltered power spectrum.}
	\label{fig:powerspectra}
\end{figure}


\subsection{Heartbeat peak detection}
In order to detect peaks of the heartbeat signals, the derivative of the signal was needed and was computed via a simple algorithm based on a backward implicit finite difference method, as depicted in Equation \eqref{eq:finite_diff}
\begin{equation}\label{eq:finite_diff}
	u'(x) = \frac{u(x) - u(x-h)}{h},
\end{equation}
where $u(x)$ is the value of the current time sample, $u(x-h)$ is the value of the previous time sample two samples ago and $h$ is the distance between each time sample, which was $1/fs$. 

To classify a peak, the following criteria were employed: the signal had to reach a threshold value equaling 40\% of the maximum signal value, and the derivative of the signal needed to lie within $\pm5\%$ of the maximum and minimum derivative values. To prevent multiple detections within a short interval, a waiting period of 200 samples was enforced. If these conditions were met, the index was classified as a peak. Given the above conditions, if the PPG signal values were to decrease significantly, subsequent peaks might fail to be detected. To correct for this potential issue, the peak threshold was reset after a duration of $2f_s$, establishing a new threshold based on the current maximum value. The algorithm was evaluated through visual inspection of the marked peaks. An interval of 10 seconds can be seen in Figure \ref{fig:peaks}.
\begin{figure}
\centering
	\includegraphics[width=\linewidth, trim={2.5cm, 1cm, 2cm, 1cm}, clip]{Figures/peaks.png}
	\caption{Evaluation of the heartbeat peak detection algorithm.}
	\label{fig:peaks}
\end{figure}
In order to evaluate the algorithm the peaks were counted manually and then compared to the number of peaks detected by it. The evaluation of the MATLAB implementation can be found in the results.

\subsection{Heart rate calculation}
The heart rate was computed during intervals of $15f_s$, where the number of peaks was divided by the number of intervals and then multiplied by $60$ in order to compute the heart rate in [beats per minute].

\subsection{Microcontroller porting}
After the peak detection algorithm was tested in MATLAB, the algorithm was programmed in C and ported onto the microcontroller. 

\subsection{Sensitivity of peak detection algorithm}
In order to get some sort of a metric to describe the performance of the implemented heartbeat peak detection algorithm, the sensitivity was calculated using Equation \eqref{eqn:accuracyEquation}

\begin{equation}
	\label{eqn:accuracyEquation}
	\text{Sensitivity} = \frac{TP}{TP+FN} \times 100 \%.
\end{equation}

where $TP$ is the number of true positives (i.e. the algorithm correctly identified a heartbeat peak) and $FN$ is the number of false negatives (i.e. the algorithm was unable to identify a heartbeat peak) \cite{trevethan_sensitivity_2017}.













\newcommand{\RNum}[1]{\uppercase\expandafter{\romannumeral #1\relax}}

\section{Results}

\setlength{\tabcolsep}{3pt}
\begin{table}[ht]
	\begin{flushleft}
		\caption{Comparison of segmentation performance between Random Forest and Atlas-based methods.}
		\begin{tabularx}{\linewidth}{Xcccc}
			\toprule
			\multirow{2}{*}{Region} & \multicolumn{2}{c}{Random Forest} & \multicolumn{2}{c}{Atlas} \\ 
			\cmidrule(lr){2-3} \cmidrule(lr){4-5}
			\midrule
			Noise & No & Yes & No & Yes \\
			Amygdala & $0.47 \pm 0.06$ & $0.47 \pm 0.06$ & $\mathbf{0.63 \pm 0.08}$ & $0.61 \pm 0.07$ \\
			Hippocampus & $0.43 \pm 0.05$ & $0.43 \pm 0.06$ & $0.61 \pm 0.10$ & $\mathbf{0.63 \pm 0.06}$ \\
			Thalamus & $0.68 \pm 0.10$ & $0.68 \pm 0.10$ & $\mathbf{0.79\pm 0.04}$ & $0.78 \pm 0.05$ \\
			Grey Matter & $0.73 \pm 0.01$ & $\mathbf{0.74 \pm 0.01}$ & $0.53 \pm 0.02 $ & $0.52 \pm 0.03$ \\
			White Matter & $\mathbf{0.83 \pm 0.02}$ & $\mathbf{0.83 \pm 0.02}$ & $0.66 \pm 0.03$ & $0.66 \pm 0.03$ \\
			Time (s)  & $197.3 \pm 6.3$  & $204.4 \pm 7.3$  & $6.0 \pm 1.8$  & $\mathbf{5.8 \pm 0.7}$ \\
			\bottomrule
		\end{tabularx}
	\end{flushleft}	
	\label{tab:performance_comparison}
\end{table}

The comparison of segmentation performance between the Random Forest-based and Atlas-based methods, both with and without noise, revealed distinct trends across brain regions. These differences can be seen in Table \RNum{1}. For the amygdala, the Atlas-based method without noise achieved the highest Dice score (0.63 ± 0.08) compared to the Random Forest method (0.47 ± 0.06). With noise, the Atlas-based method scored slightly lower (0.61 ± 0.07) but still outperformed the Random Forest method (0.47 ± 0.06). Similarly, for the hippocampus, the Atlas-based method performed better in noisy conditions, achieving the highest Dice score (0.63 ± 0.06) compared to the performance without noise (0.61 ± 0.10), while the Random Forest method showed no variability, scoring consistently (0.43 ± 0.05).
In the thalamus, the Atlas-based method achieved the best performance without noise (0.79 ± 0.04), slightly decreasing in the presence of noise (0.78 ± 0.05), whereas the Random Forest method showed consistent performance (0.68 ± 0.10) across conditions. Contrastingly, for grey matter, the Random Forest method performed better with noise (0.74 ± 0.01) than without noise (0.73 ± 0.01), while the Atlas-based method performed significantly worse, scoring 0.53 ± 0.02 without noise and 0.52 ± 0.03 with noise. For white matter, the Random Forest method outperformed the Atlas-based method with a consistent Dice score of 0.83 ± 0.02 across noise conditions, whereas the Atlas-based method achieved 0.66 ± 0.03 in both cases.
Regarding computational time, the Atlas-based method was significantly faster, taking 6.0 ± 1.8 seconds without noise and 5.8 ± 0.7 seconds with noise. In contrast, the Random Forest method required substantially more time, taking 197.3 ± 6.3 seconds without noise and 209.0 ± 7.5 seconds with noise. These results demonstrate clear differences in segmentation performance and computational efficiency between the two methods.

% !TeX encoding = utf8
% !TeX spellcheck = en_US

\section{Discussion}
The evaluation of the algorithm, as presented in the results section, reveals both its strengths and limitations in peak detection tasks. While the algorithm performed closely to manual detection with a minor discrepancy of only 2 peaks, there are notable limitations that must be addressed.

A significant limitation of the algorithm is its occasional failure to detect the true peak, instead identifying the dicrotic notch. The dicrotic notch is a secondary peak that occurs in the descending part of the pulse waveform, and mistaking this for the true peak can lead to inaccuracies in the analysis. This issue highlights a critical flaw in the algorithm's peak detection logic, which needs refinement to improve its accuracy in distinguishing between true peaks and other features of the waveform. 

This limitation could be mitigated by implementing a counter that tracks the number of samples during which the derivative exceeds a predefined threshold. If the derivative remains below this threshold, the detected feature is likely the dicrotic notch rather than a true peak. Conversely, if the derivative surpasses the threshold for a sufficient number of samples, the feature can be reliably identified as a true peak.

Furthermore, the implementation of the algorithm on a microcontroller revealed additional challenges. During this phase, the algorithm did not detect any peaks, indicating a bug in the microcontroller implementation. Due to time constraints, this bug could not be identified and resolved, preventing a thorough evaluation of the algorithm's performance on the microcontroller platform.

To further investigate the failure of the algorithm's implementation on the microcontroller, the recorded data from the microcontroller was subsequently analyzed using MATLAB. This post-hoc assessment aimed to identify the source of the implementation failure. Upon re-evaluating the algorithm in MATLAB, it was confirmed that the algorithm could indeed detect peaks accurately in this environment. This finding suggests that the core logic of the algorithm is sound. The discrepancy between the algorithm's performance in MATLAB and its failure on the microcontroller could be attributed to differences in how thresholds for the derivative were set and processed in each environment. In MATLAB, the thresholds were optimized for floating-point arithmetic, which provides high precision. However, the microcontroller operates using fixed-point arithmetic, which can introduce quantization errors and reduced precision. These differences likely resulted in the thresholds being too tight on the microcontroller, causing it to miss the zero-crossing points during sampling.

However, the exact reason for its malfunction on the microcontroller remains unresolved and requires additional investigation. The discrepancy between the MATLAB results and the microcontroller performance indicates that the issue likely lies in the implementation specifics or hardware limitations of the microcontroller platform, rather than in the algorithm itself.

Overall, while the algorithm shows promise based on its initial performance, addressing these limitations is crucial. Improving its ability to accurately detect true peaks and ensuring reliable implementation on microcontrollers will enhance its utility in practical applications. Further research and development are needed to refine the algorithm and overcome these challenges.
% !TeX encoding = utf8
% !TeX spellcheck = en_US

\section{Conclusion}
In this study, we first identified and applied suitable filters to our signal, evaluating their performance through power spectrum density analysis. We then developed an algorithm designed to detect peaks, which was subsequently implemented on a microcontroller. Despite the algorithm's promising performance in initial evaluations, the final objective of real-time peak detection to compute heart rate was not achieved due to implementation issues on the microcontroller. Further debugging and refinement are required to realize real-time heart rate computation.
 
% !TeX encoding = utf8
% !TeX spellcheck = en_US

\section{Acknowledgment}
We would like to thank Lukas Geisshüsler for guiding us throughout this project and for giving us a pleasant and instructive experience.


%----------------------------------------------------------------------------------------
%	 REFERENCES
%----------------------------------------------------------------------------------------

\printbibliography % Output the bibliography

%----------------------------------------------------------------------------------------

\end{document}
