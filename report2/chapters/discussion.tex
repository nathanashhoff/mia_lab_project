\section{Discussion}
The results presented in this study provide a comparison between the Random Forest and Atlas-based methods for brain tissue segmentation. These findings reveal that each method has its strengths and weaknesses, which are highly dependent on the target brain structures and the computational resources available.

The Random Forest method demonstrated better performance in segmenting larger structures, such as white matter and grey matter. Features capturing intensity, texture, and spatial distribution play a key role in distinguishing tissue types \cite{b15}. Larger structures like white matter tend to exhibit more uniform and distinctive intensity patterns, which are more easily captured by statistical features such as mean intensity and variance. Additionally, textural features, which quantify patterns and variations, are more consistent and pronounced in larger regions, making them easier to classify. In contrast, smaller structures, such as the thalamus, present challenges due to their limited size and less distinct textural properties. The statistical sampling of features in Random Forest models is designed to handle diverse and high-dimensional data, but smaller structures contribute fewer representative samples during the training process. This imbalance can lead to a bias in favor of larger structures that dominate the feature set. Consequently, the Random Forest method's reliance on a multidimensional feature set enables it to perform better in segmenting larger regions where spatial and intensity variations are more readily learned, while smaller regions remain more challenging to classify accurately.

The Atlas-based method showed better performance in segmenting smaller structures, such as the thalamus, which exhibit a high degree of consistency in shape, size, and position across subjects \cite{b16}. This anatomical consistency makes these structures particularly well-suited for atlas-based approaches. The compact, ellipsoid shape of the thalamus further simplifies segmentation, reducing the likelihood of errors. Additionally, the method’s reduced sensitivity to minor global misalignments ensures that even if the overall brain alignment is slightly off, the localized region around the thalamus can still align well. When the atlas registration is performed accurately, these factors enable reliable identification of the thalamus and other small structures in proximity \cite{b17}. The inherent simplicity of the atlas-based method, which does not require a complex classifier, increases its effectiveness for smaller, well-defined regions that are consistently identifiable across subjects. However Atlas-based method had difficulty segmenting grey matter, as evidenced by the lower Dice scores in this region, and also performed poorly with white matter compared to the Random Forest method. Grey matter presents a unique challenge because it is not as homogeneous as other tissue types, with complex anatomical boundaries and high variability between individuals in terms of size and shape \cite{b18}. Similarly, white matter segmentation suffers from low contrast with neighboring tissues and partial volume effects, which blur boundaries and reduce segmentation accuracy.

Additionally, the atlas-based method's reliance on the quality of the initial atlas can limit its performance for both grey and white matter. If the atlas was constructed from a population with different anatomical characteristics than the study dataset, segmentation accuracy may degrade significantly. The widespread and convoluted nature of grey matter, along with the complex organization of white matter tracts, increases sensitivity to registration errors. Misalignments during registration can lead to compounded inaccuracies, particularly for larger, less consistent structures like white matter. Furthermore, the static nature of the atlas-based method prevents it from adapting dynamically to individual anatomical variability, unlike the Random Forest method, which uses a large and diverse training dataset to flexibly address such variations and therefore produces more accurate segmentations for both grey and white matter.

In addition to segmentation performance, computational efficiency is a key consideration. The Atlas-based method was significantly faster, completing segmentation in seconds compared to the minutes required by the Random Forest method. This efficiency stems from its reliance on predefined atlases and transformations, involving only resampling and registration steps, making it ideal for large datasets. This speed advantage is critical in clinical settings, where rapid processing is often necessary \cite{b19}. In contrast, the Random Forest method’s extensive prediction steps make it more computationally demanding and less suitable for real-time applications without further optimization.

Both methods demonstrated robustness against salt and pepper noise, maintaining segmentation performance despite image disruptions. While it is expected that the Atlas-based method, relying on spatial registration, would handle noise well if the process is robust, the Random Forest method's resilience was more surprising. Notably, the Random Forest performed well on salt and pepper test images even when trained solely on normal data, showcasing its ability to generalize effectively to noisy conditions. This highlights an unexpected strength, particularly valuable in real-world applications where medical images often contain noise from artifacts or patient movement \cite{b20}.

The study has several limitations that must be considered. First, the dataset used in this study may not fully capture the variability in brain anatomy across different populations, such as those with diverse age groups or pathologies. This limitation could affect the generalizability of the findings to broader clinical settings. Second, the Atlas-based method’s performance heavily depends on accurate registration, and misalignments can significantly degrade segmentation quality, particularly in regions with high inter-subject variability. To address this, future work could involve constructing multiple atlases tailored to specific populations, allowing better capture of anatomical variations and improving segmentation accuracy. Lastly, while both methods were robust to salt and pepper noise, their resilience to other noise types, such as Gaussian or motion artifacts, was not evaluated, leaving their performance under such conditions uncertain.