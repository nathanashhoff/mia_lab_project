%\section*{References}

\begin{thebibliography}{00}
	\bibitem{b1} “What Is Dementia? Symptoms, Types, and Diagnosis,” National Institute on Aging. Accessed: Nov. 16, 2024. [Online]. Available: https://www.nia.nih.gov/health/alzheimers-and-dementia/what-dementia-symptoms-types-and-diagnosis
	
	\bibitem{b2} “Dementia.” Accessed: Nov. 16, 2024. [Online]. Available: https://www.who.int/news-room/fact-sheets/detail/dementia
	
	\bibitem{b3} A. A. Rostagno, “Pathogenesis of Alzheimer’s Disease,” IJMS, vol. 24, no. 1, p. 107, Dec. 2022, doi: 10.3390/ijms24010107.
	
	\bibitem{b4} K. Gustaw-Rothenberg et al., “Biomarkers in Alzheimer’s Disease: Past, Present and Future,” Biomarkers Med., vol. 4, no. 1, pp. 15–26, Feb. 2010, doi: 10.2217/bmm.09.86.
	
	\bibitem{b5} W. M. Van Oostveen and E. C. M. De Lange, “Imaging Techniques in Alzheimer’s Disease: A Review of Applications in Early Diagnosis and Longitudinal Monitoring,” IJMS, vol. 22, no. 4, p. 2110, Feb. 2021, doi: 10.3390/ijms22042110.
	
	\bibitem{b6} B. Foster, U. Bagci, A. Mansoor, Z. Xu, and D. J. Mollura, “A review on segmentation of positron emission tomography images,” Computers in Biology and Medicine, vol. 50, pp. 76–96, Jul. 2014, doi: 10.1016/j.compbiomed.2014.04.014.
	
	\bibitem{b7} A. W. Toga and P. M. Thompson, “Chapter 43 - Image Registration and the Construction of Multidimensional Brain Atlases,” in Handbook of Medical Image Processing and Analysis (Second Edition), I. N. Bankman, Ed., Academic Press, 2009, pp. 707–724. [Online]. Available: https://www.sciencedirect.com/science/article/pii/B9780123739049500532
	
	\bibitem{b8} C. S. Perone and J. Cohen-Adad, “Promises and limitations of deep learning for medical image segmentation,” J Med Artif Intell, vol. 2, pp. 1–1, Jan. 2019, doi: 10.21037/jmai.2019.01.01.
	
	\bibitem{b9} E. Goceri, “Challenges and Recent Solutions for Image Segmentation in the Era of Deep Learning,” in 2019 Ninth International Conference on Image Processing Theory, Tools and Applications (IPTA), Istanbul, Turkey: IEEE, Nov. 2019, pp. 1–6. doi: 10.1109/IPTA.2019.8936087.
	
	\bibitem{b10} S. Pereira, A. Pinto, J. Oliveira, A. M. Mendrik, J. H. Correia, and C. A. Silva, “Automatic brain tissue segmentation in MR images using Random Forests and Conditional Random Fields,” Journal of Neuroscience Methods, vol. 270, pp. 111–123, Sep. 2016, doi: 10.1016/j.jneumeth.2016.06.017.
	
	\bibitem{b11} “Medical Image Preprocessing.” Accessed: Nov. 30, 2024. [Online]. Available: https://ch.mathworks.com/help/medical-imaging/ug/overview-medical-image-preprocessing.html
	
	\bibitem{b12} “Medical Image Registration.” Accessed: Nov. 30, 2024. [Online]. Available: https://ch.mathworks.com/help/medical-imaging/ug/medical-image-registration.html
	
	\bibitem{b13} M. Salvi, U. R. Acharya, F. Molinari, and K. M. Meiburger, “The impact of pre- and post-image processing techniques on deep learning frameworks: A comprehensive review for digital pathology image analysis,” Computers in Biology and Medicine, vol. 128, p. 104129, Jan. 2021, doi: 10.1016/j.compbiomed.2020.104129.
	
	\bibitem{b14} “Understanding Evaluation Metrics in Medical Image Segmentation,” Medium. Accessed: Nov. 30, 2024. [Online]. Available: https://medium.com/@nghihuynh\_37300/understanding-evaluation-metrics-in-medical-image-segmentation-d289a373a3f
	

\end{thebibliography}





