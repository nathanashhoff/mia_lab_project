%\section*{References}

\begin{thebibliography}{00}
	\bibitem{b1} “What Is Dementia? Symptoms, Types, and Diagnosis,” National Institute on Aging. Accessed: Nov. 16, 2024. [Online]. Available: https://www.nia.nih.gov/health/alzheimers-and-dementia/what-dementia-symptoms-types-and-diagnosis
	
	\bibitem{b2} “Dementia.” Accessed: Nov. 16, 2024. [Online]. Available: https://www.who.int/news-room/fact-sheets/detail/dementia
	
	\bibitem{b3} A. A. Rostagno, “Pathogenesis of Alzheimer’s Disease,” IJMS, vol. 24, no. 1, p. 107, Dec. 2022, doi: 10.3390/ijms24010107.
	
	\bibitem{b4} K. Gustaw-Rothenberg et al., “Biomarkers in Alzheimer’s Disease: Past, Present and Future,” Biomarkers Med., vol. 4, no. 1, pp. 15–26, Feb. 2010, doi: 10.2217/bmm.09.86.
	
	\bibitem{b5} W. M. Van Oostveen and E. C. M. De Lange, “Imaging Techniques in Alzheimer’s Disease: A Review of Applications in Early Diagnosis and Longitudinal Monitoring,” IJMS, vol. 22, no. 4, p. 2110, Feb. 2021, doi: 10.3390/ijms22042110.
	
	\bibitem{b6} B. Foster, U. Bagci, A. Mansoor, Z. Xu, and D. J. Mollura, “A review on segmentation of positron emission tomography images,” Computers in Biology and Medicine, vol. 50, pp. 76–96, Jul. 2014, doi: 10.1016/j.compbiomed.2014.04.014.
	
	\bibitem{b7} A. W. Toga and P. M. Thompson, “Chapter 43 - Image Registration and the Construction of Multidimensional Brain Atlases,” in Handbook of Medical Image Processing and Analysis (Second Edition), I. N. Bankman, Ed., Academic Press, 2009, pp. 707–724. [Online]. Available: https://www.sciencedirect.com/science/article/pii/B9780123739049500532
	
	\bibitem{b8} C. S. Perone and J. Cohen-Adad, “Promises and limitations of deep learning for medical image segmentation,” J Med Artif Intell, vol. 2, pp. 1–1, Jan. 2019, doi: 10.21037/jmai.2019.01.01.
	
	\bibitem{b9} E. Goceri, “Challenges and Recent Solutions for Image Segmentation in the Era of Deep Learning,” in 2019 Ninth International Conference on Image Processing Theory, Tools and Applications (IPTA), Istanbul, Turkey: IEEE, Nov. 2019, pp. 1–6. doi: 10.1109/IPTA.2019.8936087.
	
	\bibitem{b10} S. Pereira, A. Pinto, J. Oliveira, A. M. Mendrik, J. H. Correia, and C. A. Silva, “Automatic brain tissue segmentation in MR images using Random Forests and Conditional Random Fields,” Journal of Neuroscience Methods, vol. 270, pp. 111–123, Sep. 2016, doi: 10.1016/j.jneumeth.2016.06.017.
	
	\bibitem{b11} “Medical Image Preprocessing.” Accessed: Nov. 30, 2024. [Online]. Available: https://ch.mathworks.com/help/medical-imaging/ug/overview-medical-image-preprocessing.html
	
	\bibitem{b12} “Medical Image Registration.” Accessed: Nov. 30, 2024. [Online]. Available: https://ch.mathworks.com/help/medical-imaging/ug/medical-image-registration.html
	
	\bibitem{b13} M. Salvi, U. R. Acharya, F. Molinari, and K. M. Meiburger, “The impact of pre- and post-image processing techniques on deep learning frameworks: A comprehensive review for digital pathology image analysis,” Computers in Biology and Medicine, vol. 128, p. 104129, Jan. 2021, doi: 10.1016/j.compbiomed.2020.104129.
	
	\bibitem{b14} “Understanding Evaluation Metrics in Medical Image Segmentation,” Medium. Accessed: Nov. 30, 2024. [Online]. Available: https://medium.com/@nghihuynh\_37300/understanding-evaluation-metrics-in-medical-image-segmentation-d289a373a3f
	
	\bibitem{b15} 
	C. Liu, R. Zhao, W. Xie, and M. Pang, "Pathological lung segmentation based on random forest combined with deep model and multi-scale superpixels," *Neural Processing Letters*, vol. 52, no. 2, pp. 1631–1649, Oct. 2020, doi: 10.1007/s11063-020-10330-8. [Online]. Available: https://doi.org/10.1007/s11063-020-10330-8
	
	\bibitem{b16} 
	N. R. Damle, T. Ikuta, M. John, B. D. Peters, P. DeRosse, A. K. Malhotra, and P. R. Szeszko, "Relationship among interthalamic adhesion size, thalamic anatomy and neuropsychological functions in healthy volunteers," *Brain Structure and Function*, vol. 222, no. 5, pp. 2183–2192, Jul. 2017, doi: 10.1007/s00429-016-1334-6. [Online]. Available: https://doi.org/10.1007/s00429-016-1334-6
	
	\bibitem{b17} 
	A. Jakab, R. Blanc, E. L. Berényi, and G. Székely, "Generation of Individualized Thalamus Target Maps by Using Statistical Shape Models and Thalamocortical Tractography," *American Journal of Neuroradiology*, vol. 33, no. 11, pp. 2110–2116, Dec. 2012, doi: 10.3174/ajnr.A3140. [Online]. Available: https://www.ajnr.org/content/33/11/2110
	
	\bibitem{b18} 
	E. Radulescu, B. Ganeshan, L. Minati, F. D. C. C. Beacher, M. A. Gray, C. Chatwin, R. C. D. Young, N. A. Harrison, and H. D. Critchley, "Gray matter textural heterogeneity as a potential in-vivo biomarker of fine structural abnormalities in Asperger syndrome," *The Pharmacogenomics Journal*, vol. 13, no. 1, pp. 70–79, Feb. 2013, doi: 10.1038/tpj.2012.3. [Online]. Available: https://www.nature.com/articles/tpj20123
	
	\bibitem{b19} 
	C. A. S. J. Gulo, A. C. Sementille, and J. M. R. S. Tavares, "Techniques of medical image processing and analysis accelerated by high-performance computing: a systematic literature review," *Journal of Real-Time Image Processing*, vol. 16, no. 6, pp. 1891–1908, Dec. 2019, doi: 10.1007/s11554-017-0734-z. [Online]. Available: https://doi.org/10.1007/s11554-017-0734-z
	
	\bibitem{b20} 
	R. R. Kumar and R. Priyadarshi, "Denoising and segmentation in medical image analysis: A comprehensive review on machine learning and deep learning approaches," *Multimedia Tools and Applications*, May 2024, doi: 10.1007/s11042-024-19313-6. [Online]. Available: https://doi.org/10.1007/s11042-024-19313-6
	

\end{thebibliography}





