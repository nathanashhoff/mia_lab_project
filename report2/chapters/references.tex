%\section*{References}

\begin{thebibliography}{00}
	\bibitem{b1} “What Is Dementia? Symptoms, Types, and Diagnosis,” National Institute on Aging. Accessed: Nov. 16, 2024. [Online]. Available: https://www.nia.nih.gov/health/alzheimers-and-dementia/what-dementia-symptoms-types-and-diagnosis
	\bibitem{b2} “Dementia.” Accessed: Nov. 16, 2024. [Online]. Available: https://www.who.int/news-room/fact-sheets/detail/dementia
	\bibitem{b3} A. A. Rostagno, “Pathogenesis of Alzheimer’s Disease,” IJMS, vol. 24, no. 1, p. 107, Dec. 2022, doi: 10.3390/ijms24010107.
	\bibitem{b4} K. Gustaw-Rothenberg et al., “Biomarkers in Alzheimer’s Disease: Past, Present and Future,” Biomarkers Med., vol. 4, no. 1, pp. 15–26, Feb. 2010, doi: 10.2217/bmm.09.86.
	\bibitem{b5} W. M. Van Oostveen and E. C. M. De Lange, “Imaging Techniques in Alzheimer’s Disease: A Review of Applications in Early Diagnosis and Longitudinal Monitoring,” IJMS, vol. 22, no. 4, p. 2110, Feb. 2021, doi: 10.3390/ijms22042110.
	\bibitem{b6} B. Foster, U. Bagci, A. Mansoor, Z. Xu, and D. J. Mollura, “A review on segmentation of positron emission tomography images,” Computers in Biology and Medicine, vol. 50, pp. 76–96, Jul. 2014, doi: 10.1016/j.compbiomed.2014.04.014.
	\bibitem{b7} [1] A. W. Toga and P. M. Thompson, “Chapter 43 - Image Registration and the Construction of Multidimensional Brain Atlases,” in Handbook of Medical Image Processing and Analysis (Second Edition), I. N. Bankman, Ed., Academic Press, 2009, pp. 707–724. [Online]. Available: https://www.sciencedirect.com/science/article/pii/B9780123739049500532
\end{thebibliography}





