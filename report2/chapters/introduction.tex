\section{Introduction}

\epigraph{Everything we do, every thought we've ever had, is produced by the human brain.}{-- Dr. Neil deGrasse Tyson}

The human brain's remarkable capacity to navigate and master the intricate complexities of our lives is nothing short of \emph{mind-blowing}\textemdash{}pun intended. Equally thought-provoking, nevertheless, is the question of why it can malfunction.

Dementia is a progressive neurodegenerative disease which results in the loss of cognitive functions such as memory, thought and reasoning skills and negatively interferes with a person's activities of daily life \cite{b1}. The World Health Organization estimates that more than 55 million people world-wide are suffering from dementia, with Alzheimer's disease being the most dominant form \cite{b2}. 

While the trigger and driving force behind the progression of Alzheimer's disease still remain uncertain, the neuropathological hallmarks of the disease are the selective loss of cortical neurons within the hippocampus, the temporal lobe and frontal lobe, the gradual loss of synapses, the deposition of amyloid-\textbeta\ forming senile plaques and the presence of intraneuronal neurofibrillary tangles which contain highly phosphorylated microtubule-associated tau proteins. \cite{b3,b4} 

Current assessment of Alzheimer's disease is based on a qualitative psychometric clinical diagnosis relying on criteria established by the Diagnostic and Statistical Manual of Mental Disorders and the National Institute of Neurological and Communicative Disorders and Stroke/Alzheimer's Disease and Related Disorders Association with a definitive diagnosis only confirmed post-mortem through histological staining of brain tissue \cite{b4,b5}. Considering the wealth of modern scientific advancements, these methods have become rather antiquated and frequently overlook an early diagnosis, crucial for any potential intervention to effectively delay the disease process. Hence the ongoing search for a robust biomarker which can identify, assess and monitor the unfolding of Alzheimer's disease \cite{b4}.

While molecular biomarkers, requiring a blood or cerebrospinal fluid sample, have been proposed, morphological biomarkers, such as hippocampus volume and cortical thinning obtained from magnetic resonance imaging or brain hypometabolism information obtained from positron emission tomography, are also promising \cite{b4,b5}. To leverage and evaluate these biomarkers, the regions of interests must be extracted from the scan, a process called image segmentation. Manual segmentation is the simplest technique but suffers many drawbacks such as being time consuming, labor intensive and operator-dependent rendering it ill-suited for routine clinical practice \cite{b6}. Consequently, current research is investigating the use of automated methods.

One such method is atlas-based segmentation. A brain atlas is a map that shows the typical structure of the brain anatomy as it is built from population-based imaging data after registration, warping and overlay to a common reference frame and is based on the assumption that all brains, to a degree of deformation, resemble a prototypical template. Applying this concept of an atlas to segmentation tasks leads to label-based approaches, or probabilistic anatomy maps, in which manual anatomical segmentation labels from a library are mapped to the atlas image \cite{b7}. The registration from atlas space to the coordinate system of a newly-obtained brain image then automatically generates with it the tailored segmentation labels.

Of interest in this research study is the performance comparison between an altas-based and machine-learning based segmentation of several brain regions\textemdash{}the white matter, the gray matter, the hippocampus, the amygdala and the thalamus. Machine learning approaches carry several associated costs such as high computational demands, potentially slow inference times, lack of interpretability and transparency that pose a challenge in clinical adoption, potential to overfit and consequent ineffectiveness on unseen data, and dependency on preprocessing, such as skull stripping, bias field correction and normalization which can introduce additional sources for error. Accordingly, the motivation for the juxtaposition in this study is to evaluate whether a so-called more simplistic atlas-based approach can match or even outperform an esoteric and not without drawbacks machine learning approach, calling into question whether the more sophisticated approach is sufficiently warranted for brain tissue segmentation.



















