\newcommand{\RNum}[1]{\uppercase\expandafter{\romannumeral #1\relax}}

\section{Results}

\setlength{\tabcolsep}{3pt}
\begin{table}[ht]
	\begin{flushleft}
		\caption{Comparison of Dice similarity coefficients between the Random Forest and Atlas-based methods.}
		\begin{tabularx}{\linewidth}{Xcccc}
			\toprule
			\multirow{2}{*}{Region} & \multicolumn{2}{c}{Random Forest} & \multicolumn{2}{c}{Atlas} \\ 
			\cmidrule(lr){2-3} \cmidrule(lr){4-5}
			\midrule
			Noise & No & Yes & No & Yes \\
			Amygdala & $0.47 \pm 0.06$ & $0.47 \pm 0.06$ & $\mathbf{0.63 \pm 0.08}$ & $0.61 \pm 0.07$ \\
			Hippocampus & $0.43 \pm 0.05$ & $0.43 \pm 0.06$ & $0.61 \pm 0.10$ & $\mathbf{0.63 \pm 0.06}$ \\
			Thalamus & $0.68 \pm 0.10$ & $0.68 \pm 0.10$ & $\mathbf{0.79\pm 0.04}$ & $0.78 \pm 0.05$ \\
			Grey Matter & $0.73 \pm 0.01$ & $\mathbf{0.74 \pm 0.01}$ & $0.53 \pm 0.02 $ & $0.52 \pm 0.03$ \\
			White Matter & $\mathbf{0.83 \pm 0.02}$ & $\mathbf{0.83 \pm 0.02}$ & $0.66 \pm 0.03$ & $0.66 \pm 0.03$ \\
			Time (s)  & $197.3 \pm 6.3$  & $204.4 \pm 7.3$  & $6.0 \pm 1.8$  & $\mathbf{5.8 \pm 0.7}$ \\
			\bottomrule
		\end{tabularx}
	\end{flushleft}	
	%\vspace{0.1em} % Optional: Add some spacing
	\scriptsize{\textit{These are mean and standard deviation values across all 10 patients from the test data.}}
	\label{tab:performance_comparison}
\end{table}

The mean Dice similarity coefficients along with their standard deviations for the various brain regions across all 10 test patients using the Random Forest and atlas-based methods can be seen in Table \RNum{1}. The mean computational times along with their standard deviations have also been included.

Generally speaking, for both methods, the obtained mean Dice similarity coefficients remain relatively the same for both the original and simulated noisy images. Therefore, it seems that this simulated noise had very little effect on the performance of the models.

For the amygdala, hippocampus and thalamus, the atlas-based method performed better than the Random Forest model as we obtained, respectively, mean Dice similarity coefficients of 0.63, 0.61 and 0.79 compared to 0.47, 0.43 and 0.68.

For the grey matter and white matter, the Random Forest model performed better than the atlas-based method as we obtained, respectively, mean Dice similarity coefficients of 0.73 and 0.83 compared to 0.53 and 0.66.

%For a qualitative comparison of the resulting segmentation labels, the reader is invited to consult the appendix, where specific patients have been selected and displayed.

Regarding computational time, the atlas-based method was significantly faster, taking 6.0 ± 1.8 seconds without noise and 5.8 ± 0.7 seconds with noise while the Random Forest method required substantially more time, taking 197.3 ± 6.3 seconds without noise and 209.0 ± 7.5 seconds with noise.

\begin{comment}
The comparison of segmentation performance between the Random Forest-based and Atlas-based methods, both with and without noise, revealed distinct trends across brain regions. These differences can be seen in Table \RNum{1}. 

For the amygdala, the Atlas-based method without noise achieved the highest Dice score (0.63 ± 0.08) compared to the Random Forest method (0.47 ± 0.06). With noise, the Atlas-based method scored slightly lower (0.61 ± 0.07) but still outperformed the Random Forest method (0.47 ± 0.06). 

For the hippocampus, the Atlas-based method performed better in noisy conditions, achieving the highest Dice score (0.63 ± 0.06) compared to the performance without noise (0.61 ± 0.10), while the Random Forest method showed no variability, scoring consistently (0.43 ± 0.05).

For the thalamus, the Atlas-based method achieved the best performance without noise (0.79 ± 0.04), slightly decreasing in the presence of noise (0.78 ± 0.05), whereas the Random Forest method showed consistent performance (0.68 ± 0.10) across conditions. 

For grey matter, the Random Forest method performed better with noise (0.74 ± 0.01) than without noise (0.73 ± 0.01), while the Atlas-based method performed significantly worse, scoring 0.53 ± 0.02 without noise and 0.52 ± 0.03 with noise. For white matter, the Random Forest method outperformed the Atlas-based method with a consistent Dice score of 0.83 ± 0.02 across noise conditions, whereas the Atlas-based method achieved 0.66 ± 0.03 in both cases.

Regarding computational time, the Atlas-based method was significantly faster, taking 6.0 ± 1.8 seconds without noise and 5.8 ± 0.7 seconds with noise. In contrast, the Random Forest method required substantially more time, taking 197.3 ± 6.3 seconds without noise and 209.0 ± 7.5 seconds with noise. These results demonstrate clear differences in segmentation performance and computational efficiency between the two methods.
\end{comment}