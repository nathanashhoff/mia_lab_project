% Magic comments for TeXStudio
% !TeX program = pdflatex
% !BIB program = biber
% !TeX encoding = utf8
% !TeX spellcheck = en_US

\documentclass[conference]{IEEEtran}

% It doesn't seem like we need this
% BibLaTeX bibliography file
%\bibliography{bibAntoine.bib} 
%\bibliography{bibNathan.bib} 
%\bibliography{bibSandro.bib}

% I added
\usepackage{epigraph}
\usepackage{textgreek}

\IEEEoverridecommandlockouts
% The preceding line is only needed to identify funding in the first footnote. If that is unneeded, please comment it out.
\usepackage{cite}
\renewcommand\citepunct{,}
\usepackage{float}
\usepackage{comment}
\usepackage{amsmath,amssymb,amsfonts}
\usepackage{algorithmic}
\usepackage{graphicx}
\usepackage{textcomp}
\usepackage{xcolor}
\usepackage{booktabs}
\usepackage{multirow}
\usepackage{tabularx}
\def\BibTeX{{\rm B\kern-.05em{\sc i\kern-.025em b}\kern-.08em
    T\kern-.1667em\lower.7ex\hbox{E}\kern-.125emX}}

\begin{document}

\title{Atlas-based Segmentation Consists of a Powerful Baseline for Brain Tissue Segmentation when Compared to a ML-based Approach\\
\thanks{All authors declare that they have no conflicts of interest.}
}

\begin{comment}
\author{\IEEEauthorblockN{Antoine Biebuyck}
\IEEEauthorblockA{\textit{ARTORG Center for BME Research} \\
\textit{Université de Berne}\\
Berne, Suisse \\
antoine.biebuyck@students.unibe.ch}
\and
\IEEEauthorblockN{Nathan Hoffman}
\IEEEauthorblockA{\textit{ARTORG Center for BME Research} \\
\textit{Universität Bern}\\
Bern, Schweiz \\
nathan.hoffman@students.unibe.ch}
\and
\IEEEauthorblockN{Sandro Scherrer}
\IEEEauthorblockA{\textit{ARTORG Center for BME Research} \\
\textit{Università di Berna}\\
Berna, Svizzera \\
sandro.scherrer@students.unibe.ch}
}
\end{comment}

\author{
	\IEEEauthorblockN{Antoine Biebuyck\textsuperscript{1}, Nathan Hoffman\textsuperscript{1}, Sandro Scherrer\textsuperscript{1}}
	\IEEEauthorblockA{\textsuperscript{1}ARTORG Center for Biomedical Engineering Research, University of Bern, Bern, Switzerland \\
		Emails: antoine.biebuyck@students.unibe.ch, nathan.hoffman@students.unibe.ch, sandro.scherrer@students.unibe.ch}
}

\maketitle

\begin{abstract}
Alzheimer's disease is on the rise and presents a threat to our aging populations. Reliable biomarkers are required to permit early detection and comprehensive monitoring. One such class of biomarkers arise from volumetric and structural information obtained from magnetic resonance imaging sequences. However, such characterizations rely on precise segmentation which, when performed manually, is too resource-heavy to become adopted in routine clinical workflows. Therefore, automatic segmentation processes are essential to investigate. In this study, we are interested in comparing the performance of an atlas-based method to a machine learning Random Forest model, to evaluate whether the former can match or outperform the latter. Our findings suggest that it depends, in that, the atlas-based method performed better on the smaller brain structures whereas the Random Forest model performed better on the larger brain structures.
\end{abstract}

\begin{IEEEkeywords}
Alzheimer's disease, Magnetic resonance imaging, Image segmentation, Machine learning, Atlas
\end{IEEEkeywords}

\section{Introduction}

\epigraph{What is real? How do you define 'real'? If you're talking about what you can feel, what you can smell, what you can taste and see, then "real" is simply electrical signals interpreted by your brain.}{-- Morpheus, The Matrix}

%\epigraph{Everything we do, every thought we've ever had, is produced by the human brain.}{-- Dr. Neil deGrasse Tyson}

The human brain's remarkable capacity to navigate and master the intricate complexities of our lives is nothing short of \emph{mind-blowing}\textemdash{}pun intended. Equally thought-provoking, nevertheless, is the question of why it can malfunction.

Dementia is a progressive neurodegenerative disease which results in the loss of cognitive functions such as memory, thought and reasoning skills and negatively interferes with a person's activities of daily life \cite{b1}. The World Health Organization estimates that more than 55 million people world-wide are suffering from dementia, with Alzheimer's disease being the most dominant form \cite{b2}. 

While the trigger and driving force behind the progression of Alzheimer's disease still remain uncertain, the neuropathological hallmarks of the disease are the selective loss of cortical neurons within the hippocampus, the temporal lobe and frontal lobe, the gradual loss of synapses, the deposition of amyloid-\textbeta\ forming senile plaques and the presence of intraneuronal neurofibrillary tangles which contain highly phosphorylated microtubule-associated tau proteins. \cite{b3,b4} 

Current assessment of Alzheimer's disease is based on a qualitative psychometric clinical diagnosis relying on criteria established by the Diagnostic and Statistical Manual of Mental Disorders and the National Institute of Neurological and Communicative Disorders and Stroke/Alzheimer's Disease and Related Disorders Association with a definitive diagnosis only confirmed post-mortem through histological staining of brain tissue \cite{b4,b5}. Considering the wealth of modern scientific advancements, these methods have become rather antiquated and frequently overlook an early diagnosis, crucial for any potential intervention to effectively delay the disease process. Hence the ongoing search for a robust biomarker which can identify, assess and monitor the unfolding of Alzheimer's disease \cite{b4}.

While molecular biomarkers, requiring a blood or cerebrospinal fluid sample, have been proposed, morphological biomarkers, such as hippocampus volume and cortical thinning obtained from magnetic resonance imaging or brain hypometabolism information obtained from positron emission tomography, are also promising \cite{b4,b5}. To leverage and evaluate these biomarkers, the regions of interests must be extracted from the scan, a process called image segmentation. Manual segmentation is the simplest technique but suffers from drawbacks which include the need for trained professionals, being time- and labor-intensive and operator-dependency rendering it ill-suited for routine clinical practice \cite{b6}. Consequently, current research is investigating the use of automated methods.

One such method is atlas-based segmentation. A brain atlas is a map that shows the typical structure of the brain anatomy as it is built from population-based imaging data after registration, warping and overlay to a common reference frame and is based on the assumption that all brains, to a degree of deformation, resemble a prototypical template. Applying this concept of an atlas to segmentation tasks leads to label-based approaches, or probabilistic anatomy maps, in which manual anatomical segmentation labels from a library are mapped to the atlas image \cite{b7}. The registration from atlas space to the coordinate system of a newly-obtained brain image then automatically generates with it the tailored segmentation labels.

Of interest in this research study is the performance comparison between an altas-based and machine-learning based segmentation of several brain regions\textemdash{}the white matter, the gray matter, the hippocampus, the amygdala and the thalamus. Machine learning approaches carry several associated costs such as high computational demands, potentially slow inference times, lack of interpretability and transparency that pose a challenge in clinical adoption, potential to overfit and consequent ineffectiveness on unseen data, and dependency on preprocessing, such as skull stripping, bias field correction and normalization which can introduce additional sources for error \cite{b8,b9} Accordingly, the motivation for the juxtaposition in this study is to evaluate whether a so-called more simplistic atlas-based approach can match or even outperform an esoteric and not without drawbacks machine learning approach, calling into question whether the more sophisticated approach is sufficiently warranted for brain tissue segmentation.




















\section{Materials and Methods}

In medical image analysis, a typical workflow involves executing several sequential algorithmic steps, a process collectively referred to as a pipeline \cite{b8}. The pipeline in this study consists of pre-processing, registration, feature extraction, classification, post-processing and evaluation. 

Regarding the experimental data, we utilized 20 training sets and 10 test sets of T1-weighted (T1w) and T2-weighted (T2w) MRI sequences, each accompanied by ground truth segmentation labels, brain masks and an affine transformation matrix to a provided reference atlas.

\subsection{Pre-processing}

As an initial step, pre-processing functions to refine and standardize all images in a dataset. Some common actions include background removal, noise reduction, intensity normalization and resampling \cite{b9}.

In our implementation, the pipeline began with intensity normalization, executed via Z-score normalization. This method computed the mean and standard deviation of pixel intensities, standardizing them to a zero-mean, unit-variance distribution. This step mitigated inconsistencies across the dataset. Next, skull stripping was performed, which applied binary masks to remove non-brain regions. This ensured only relevant anatomical structures were retained.

Additionally, we introduced artificial salt-and-pepper noise. This step was not part of standard pre-processing but was incorporated to test model robustness under lower image quality conditions.

% Ex: For the machine-learning approach, we proceeded with a skull stripping. 
% Ex: As an additional experimental feature, we implemented a sort of reverse pre-processing and added salt & peppper noise to our images to evaluate our models on noisy data.

\subsection{Registration}

Registration is used to align multiple images to a common reference frame. When only rotations and translations are involved, it is referred to as a rigid transformation while, when scale and skew factors are also included, it is referred to as an affine transformation. Nonlinear, deformable transformations also exist \cite{b10}. Since the corresponding affine transformations were provided in the dataset, the registration step was skipped during the Random Forest training.
% Ex: In this study, we aligned the T1- and T2-weighted images. 
% Ex: Registration was also used while constructing the atlas but this is discussed in an ensuing section.

\subsection{Feature Extraction}

Any extractable characteristic or property that describes the underlying medical image and is used in the analysis is coined as a feature. Some examples of image features are intensity, shape, and texture information \cite{b8}. In this study, we focused on extracting features that encapsulate spatial, textural, and statistical properties from the T1w and T2w MRI sequences.

Textural features, derived from local intensity distributions, were used to capture patterns and variations within tissue regions. These included metrics such as mean intensity, variance, skewness, and entropy, which are particularly valuable for distinguishing between homogeneous and heterogeneous structures. These features enable the identification of subtle differences in tissue composition. Statistical sampling of voxel intensities provided additional insights by focusing on representative subsets of the data. By generating masks to ensure balanced sampling across different labels, we accounted for variations in tissue representation and minimized the impact of class imbalance. This approach not only improved the quality of the feature set but also enhanced the training process for the Random Forest model.

%The combination of these features ensured a robust and multidimensional characterization of the images, facilitating effective downstream processing and analysis.

\subsection{Classification}

Classification is the core of the automated segmentation process, where the chosen machine learning algorithm determines the label for each voxel in the image \cite{b8}. In this study, a Random Forest classifier was utilized due to its robustness in handling high-dimensional data and its resistance to overfitting.

To optimize the classifier’s performance, a grid search was conducted to determine the best hyperparameters for both the original and the simulated noisy datasets. Interestingly, the same optimal hyperparameters were identified for both conditions, indicating the classifier’s adaptability to different data qualities. The model achieved the best results with 700 decision trees, a maximum tree depth of 45, and a square root selection of features at each split.

The Random Forest was trained on features extracted from the pre-processed images, and its predictions on test data provided voxel-level probabilities for each label. These outputs formed the basis for the subsequent evaluation step.


\subsection{Post-processing}

In this study, post-processing was not applied as it was not necessary to evaluate our hypothesis. The results were directly assessed to determine the segmentation performance, ensuring that the findings reflected the core methodologies without additional modifications.

% \# TODO: What did we do to ML results? What did we do to the atlas?
%Should we maybe just remove this subsection because we dont use any postprocessing?

\subsection{Evaluation}

The final step of the pipeline is to quantitatively assess its performance. Such a numeric score is a quick and easy way to compare and contrast different pipelines. Note that qualitative assessment of the resulting segmentation results is also very important to consider when deciding the clinically "better" result.

Dice similarity coefficients quantify the segmentation quality for each tissue type (e.g., gray matter, white matter) \cite{b12}. Evaluator tools calculate subject-wise and aggregated statistics, including mean and standard deviation, across test samples. Results are logged and saved in timestamped directories for analysis.

\subsection{Construction of Atlas Labels}
A reference atlas serves as a common space for alignment. All training images were used for generating the atlas labels by registering the segmentation labels of each patient to the provided reference atlas using the given affine transformations. The atlas labels are constructed by averaging all the individual patient labels from the training set and then assigning each voxel the label that occurs most frequently. Additionally, morphological operations, such as median filtering and Opening and Closing, were applied to refine segmentation quality and eliminate small artifacts.





\newcommand{\RNum}[1]{\uppercase\expandafter{\romannumeral #1\relax}}

\section{Results}

\setlength{\tabcolsep}{3pt}
\begin{table}[ht]
	\begin{flushleft}
		\caption{Comparison of segmentation performance between Random Forest and Atlas-based methods.}
		\begin{tabularx}{\linewidth}{Xcccc}
			\toprule
			\multirow{2}{*}{Region} & \multicolumn{2}{c}{Random Forest} & \multicolumn{2}{c}{Atlas} \\ 
			\cmidrule(lr){2-3} \cmidrule(lr){4-5}
			\midrule
			Noise & No & Yes & No & Yes \\
			Amygdala & $0.47 \pm 0.06$ & $0.47 \pm 0.06$ & $\mathbf{0.63 \pm 0.08}$ & $0.61 \pm 0.07$ \\
			Hippocampus & $0.43 \pm 0.05$ & $0.43 \pm 0.06$ & $0.61 \pm 0.10$ & $\mathbf{0.63 \pm 0.06}$ \\
			Thalamus & $0.68 \pm 0.10$ & $0.68 \pm 0.10$ & $\mathbf{0.79\pm 0.04}$ & $0.78 \pm 0.05$ \\
			Grey Matter & $0.73 \pm 0.01$ & $\mathbf{0.74 \pm 0.01}$ & $0.53 \pm 0.02 $ & $0.52 \pm 0.03$ \\
			White Matter & $\mathbf{0.83 \pm 0.02}$ & $\mathbf{0.83 \pm 0.02}$ & $0.66 \pm 0.03$ & $0.66 \pm 0.03$ \\
			Time (s)  & $197.3 \pm 6.3$  & $204.4 \pm 7.3$  & $6.0 \pm 1.8$  & $\mathbf{5.8 \pm 0.7}$ \\
			\bottomrule
		\end{tabularx}
	\end{flushleft}	
	\label{tab:performance_comparison}
\end{table}

The comparison of segmentation performance between the Random Forest-based and Atlas-based methods, both with and without noise, revealed distinct trends across brain regions. These differences can be seen in Table \RNum{1}. For the amygdala, the Atlas-based method without noise achieved the highest Dice score (0.63 ± 0.08) compared to the Random Forest method (0.47 ± 0.06). With noise, the Atlas-based method scored slightly lower (0.61 ± 0.07) but still outperformed the Random Forest method (0.47 ± 0.06). Similarly, for the hippocampus, the Atlas-based method performed better in noisy conditions, achieving the highest Dice score (0.63 ± 0.06) compared to the performance without noise (0.61 ± 0.10), while the Random Forest method showed no variability, scoring consistently (0.43 ± 0.05).
In the thalamus, the Atlas-based method achieved the best performance without noise (0.79 ± 0.04), slightly decreasing in the presence of noise (0.78 ± 0.05), whereas the Random Forest method showed consistent performance (0.68 ± 0.10) across conditions. Contrastingly, for grey matter, the Random Forest method performed better with noise (0.74 ± 0.01) than without noise (0.73 ± 0.01), while the Atlas-based method performed significantly worse, scoring 0.53 ± 0.02 without noise and 0.52 ± 0.03 with noise. For white matter, the Random Forest method outperformed the Atlas-based method with a consistent Dice score of 0.83 ± 0.02 across noise conditions, whereas the Atlas-based method achieved 0.66 ± 0.03 in both cases.
Regarding computational time, the Atlas-based method was significantly faster, taking 6.0 ± 1.8 seconds without noise and 5.8 ± 0.7 seconds with noise. In contrast, the Random Forest method required substantially more time, taking 197.3 ± 6.3 seconds without noise and 209.0 ± 7.5 seconds with noise. These results demonstrate clear differences in segmentation performance and computational efficiency between the two methods.

% !TeX encoding = utf8
% !TeX spellcheck = en_US

\section{Discussion}
The evaluation of the algorithm, as presented in the results section, reveals both its strengths and limitations in peak detection tasks. While the algorithm performed closely to manual detection with a minor discrepancy of only 2 peaks, there are notable limitations that must be addressed.

A significant limitation of the algorithm is its occasional failure to detect the true peak, instead identifying the dicrotic notch. The dicrotic notch is a secondary peak that occurs in the descending part of the pulse waveform, and mistaking this for the true peak can lead to inaccuracies in the analysis. This issue highlights a critical flaw in the algorithm's peak detection logic, which needs refinement to improve its accuracy in distinguishing between true peaks and other features of the waveform. 

This limitation could be mitigated by implementing a counter that tracks the number of samples during which the derivative exceeds a predefined threshold. If the derivative remains below this threshold, the detected feature is likely the dicrotic notch rather than a true peak. Conversely, if the derivative surpasses the threshold for a sufficient number of samples, the feature can be reliably identified as a true peak.

Furthermore, the implementation of the algorithm on a microcontroller revealed additional challenges. During this phase, the algorithm did not detect any peaks, indicating a bug in the microcontroller implementation. Due to time constraints, this bug could not be identified and resolved, preventing a thorough evaluation of the algorithm's performance on the microcontroller platform.

To further investigate the failure of the algorithm's implementation on the microcontroller, the recorded data from the microcontroller was subsequently analyzed using MATLAB. This post-hoc assessment aimed to identify the source of the implementation failure. Upon re-evaluating the algorithm in MATLAB, it was confirmed that the algorithm could indeed detect peaks accurately in this environment. This finding suggests that the core logic of the algorithm is sound. The discrepancy between the algorithm's performance in MATLAB and its failure on the microcontroller could be attributed to differences in how thresholds for the derivative were set and processed in each environment. In MATLAB, the thresholds were optimized for floating-point arithmetic, which provides high precision. However, the microcontroller operates using fixed-point arithmetic, which can introduce quantization errors and reduced precision. These differences likely resulted in the thresholds being too tight on the microcontroller, causing it to miss the zero-crossing points during sampling.

However, the exact reason for its malfunction on the microcontroller remains unresolved and requires additional investigation. The discrepancy between the MATLAB results and the microcontroller performance indicates that the issue likely lies in the implementation specifics or hardware limitations of the microcontroller platform, rather than in the algorithm itself.

Overall, while the algorithm shows promise based on its initial performance, addressing these limitations is crucial. Improving its ability to accurately detect true peaks and ensuring reliable implementation on microcontrollers will enhance its utility in practical applications. Further research and development are needed to refine the algorithm and overcome these challenges.
% !TeX encoding = utf8
% !TeX spellcheck = en_US

\section{Conclusion}
In this study, we first identified and applied suitable filters to our signal, evaluating their performance through power spectrum density analysis. We then developed an algorithm designed to detect peaks, which was subsequently implemented on a microcontroller. Despite the algorithm's promising performance in initial evaluations, the final objective of real-time peak detection to compute heart rate was not achieved due to implementation issues on the microcontroller. Further debugging and refinement are required to realize real-time heart rate computation.
 
% !TeX encoding = utf8
% !TeX spellcheck = en_US

\section{Acknowledgment}
We would like to thank Lukas Geisshüsler for guiding us throughout this project and for giving us a pleasant and instructive experience.

%\section*{References}

\begin{thebibliography}{00}
	\bibitem{b1} “What Is Dementia? Symptoms, Types, and Diagnosis,” National Institute on Aging. Accessed: Nov. 16, 2024. [Online]. Available: https://www.nia.nih.gov/health/alzheimers-and-dementia/what-dementia-symptoms-types-and-diagnosis
	\bibitem{b2} “Dementia.” Accessed: Nov. 16, 2024. [Online]. Available: https://www.who.int/news-room/fact-sheets/detail/dementia
	\bibitem{b3} A. A. Rostagno, “Pathogenesis of Alzheimer’s Disease,” IJMS, vol. 24, no. 1, p. 107, Dec. 2022, doi: 10.3390/ijms24010107.
	\bibitem{b4} K. Gustaw-Rothenberg et al., “Biomarkers in Alzheimer’s Disease: Past, Present and Future,” Biomarkers Med., vol. 4, no. 1, pp. 15–26, Feb. 2010, doi: 10.2217/bmm.09.86.
	\bibitem{b5} W. M. Van Oostveen and E. C. M. De Lange, “Imaging Techniques in Alzheimer’s Disease: A Review of Applications in Early Diagnosis and Longitudinal Monitoring,” IJMS, vol. 22, no. 4, p. 2110, Feb. 2021, doi: 10.3390/ijms22042110.
	\bibitem{b6} B. Foster, U. Bagci, A. Mansoor, Z. Xu, and D. J. Mollura, “A review on segmentation of positron emission tomography images,” Computers in Biology and Medicine, vol. 50, pp. 76–96, Jul. 2014, doi: 10.1016/j.compbiomed.2014.04.014.
	\bibitem{b7} [1] A. W. Toga and P. M. Thompson, “Chapter 43 - Image Registration and the Construction of Multidimensional Brain Atlases,” in Handbook of Medical Image Processing and Analysis (Second Edition), I. N. Bankman, Ed., Academic Press, 2009, pp. 707–724. [Online]. Available: https://www.sciencedirect.com/science/article/pii/B9780123739049500532
\end{thebibliography}







% Start of annex/appendices
%\clearpage
%\newpage
%\onecolumn % Switch to single-column mode
%\appendix
%\section{Appendix}
%\begin{figure} [H]
	\centering
	\includegraphics[width=\textwidth]{"figures/atlas before median filter"}
	\caption{Atlas Labels before Morphological Filters.}
	\label{fig:atlas-before-median-filter}
\end{figure}

\begin{figure} [H]
	\centering
	\includegraphics[width=\textwidth]{"figures/atlas after filter"}
	\caption{Atlas Labels after Morphological Filters.}
	\label{fig:atlas-after-median-filter}
\end{figure}

\begin{figure} [H]
	\centering
	\includegraphics[width=\textwidth]{"figures/atlas_best"}
	\caption{Best Overall Performing Atlas Segmentation Labels, Patient 123117.}
	\label{fig:atlas-best}
\end{figure}

\begin{figure} [H]
	\centering
	\includegraphics[width=\textwidth]{"figures/atlas_worst"}
	\caption{Worst Overall Performing Atlas-based Segmentation Labels, Patient 124422.}
	\label{fig:atlas-worst}
\end{figure}

\begin{figure} [H]
	\centering
	\includegraphics[width=\textwidth]{"figures/rf_best"}
	\caption{Best Overall Performing Random Forest Segmentation Labels, Patient 124422.}
	\label{fig:rf-best}
\end{figure}

\begin{figure} [H]
	\centering
	\includegraphics[width=\textwidth]{"figures/rf_worst"}
	\caption{Worst Overall Performing Random Forest Segmentation Labels, Patient 123117.}
	\label{fig:rf-worst}
\end{figure}




\end{document}
