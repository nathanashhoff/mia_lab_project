% Magic comments for TeXStudio
% !TeX program = pdflatex
% !BIB program = biber
% !TeX encoding = utf8
% !TeX spellcheck = en_US

\documentclass[conference]{IEEEtran}

% It doesn't seem like we need this
% BibLaTeX bibliography file
%\bibliography{bibAntoine.bib} 
%\bibliography{bibNathan.bib} 
%\bibliography{bibSandro.bib}

% I added
\usepackage{epigraph}
\usepackage{textgreek}

\IEEEoverridecommandlockouts
% The preceding line is only needed to identify funding in the first footnote. If that is unneeded, please comment it out.
\usepackage{cite}
\renewcommand\citepunct{,}
\usepackage{float}
\usepackage{comment}
\usepackage{amsmath,amssymb,amsfonts}
\usepackage{algorithmic}
\usepackage{graphicx}
\usepackage{textcomp}
\usepackage{xcolor}
\usepackage{booktabs}
\usepackage{multirow}
\usepackage{tabularx}
\def\BibTeX{{\rm B\kern-.05em{\sc i\kern-.025em b}\kern-.08em
    T\kern-.1667em\lower.7ex\hbox{E}\kern-.125emX}}

\begin{document}

\title{Atlas-based Segmentation Consists of a Powerful Baseline for Brain Tissue Segmentation when Compared to a ML-based Approach\\
\thanks{All authors declare that they have no conflicts of interest.}
}

\begin{comment}
\author{\IEEEauthorblockN{Antoine Biebuyck}
\IEEEauthorblockA{\textit{ARTORG Center for BME Research} \\
\textit{Université de Berne}\\
Berne, Suisse \\
antoine.biebuyck@students.unibe.ch}
\and
\IEEEauthorblockN{Nathan Hoffman}
\IEEEauthorblockA{\textit{ARTORG Center for BME Research} \\
\textit{Universität Bern}\\
Bern, Schweiz \\
nathan.hoffman@students.unibe.ch}
\and
\IEEEauthorblockN{Sandro Scherrer}
\IEEEauthorblockA{\textit{ARTORG Center for BME Research} \\
\textit{Università di Berna}\\
Berna, Svizzera \\
sandro.scherrer@students.unibe.ch}
}
\end{comment}

\author{
	\IEEEauthorblockN{Antoine Biebuyck\textsuperscript{1}, Nathan Hoffman\textsuperscript{1}, Sandro Scherrer\textsuperscript{1}}
	\IEEEauthorblockA{\textsuperscript{1}ARTORG Center for Biomedical Engineering Research, University of Bern, Bern, Switzerland \\
		Emails: antoine.biebuyck@students.unibe.ch, nathan.hoffman@students.unibe.ch, sandro.scherrer@students.unibe.ch}
}

\maketitle

\begin{abstract}
Alzheimer's disease is on the rise and presents a threat to our aging populations. Reliable biomarkers are required to permit early detection and comprehensive monitoring. One such class of biomarkers arise from volumetric and structural information obtained from magnetic resonance imaging sequences. However, such characterizations rely on precise segmentation which, when performed manually, is too resource-heavy to become adopted in routine clinical workflows. Therefore, automatic segmentation processes are essential to investigate. In this study, we are interested in comparing the performance of an atlas-based method to a machine learning Random Forest model, to evaluate whether the former can match or outperform the latter. Our findings suggest that it depends, in that, the atlas-based method performed better on the smaller brain structures whereas the Random Forest model performed better on the larger brain structures.
\end{abstract}

\begin{IEEEkeywords}
Alzheimer's disease, Magnetic resonance imaging, Image segmentation, Machine learning, Atlas
\end{IEEEkeywords}

% !TeX encoding = utf8
% !TeX spellcheck = en_US

\section{Introduction}

Lub dub, lub dub, lub dub... Over the course of life, the heart perpetually beats and yet one does not perceive the thump. Nevertheless, the heart beat dictates the rhythm of life. Throughout history, ancient civilizations understood the heart's fundamental force in the preservation of life. Hence, doctors in antiquity had already started to palpate the pulse to extract the rate. \cite{hajar_pulse_2018}

At the present time, palpation of the peripheral arteries still remains a customary clinical technique but several more advanced technologies have also emerged. These include electrocardiography, which measures the electrical activity of the myocardial pathway, magnetic induction tomography, which measures changes of tissue connectivity and impedance, ballistocardiography, which measures the vibrations of the body and phonocardiography, which measures the noise produced by the heart. \cite{ludwig_measurement_2018}

Despite the respective abilities of the above instrumentation to accurately measure heart rate, they possess several disadvantages, particularly concerning their integration into wearable devices. For example, electrocardiography has a complicated procedure involving twelve electrodes positioned at specific locations, consequently requiring the expertise of a trained professional. Furthermore, magnetic inductance tomography and ballistocardiography are adversly affected by movements and muscular activity while phonocardiography is impaired by surrounding noise. \cite{ludwig_measurement_2018}

In light of these shortcomings, photoplethysmography (PPG), a non-invasive optical technology, presents itself as an accessible, convenient and reliable gadget for continuous monitoring of vital signs in wearable devices \cite{kim_photoplethysmography_2023}. Photoplethysmography is capable of quantifying diverse health metrics such as blood oxygenation, arterial stiffness and blood pressure besides heart rate. Photoplethysmography operates by detecting fluctuations in light intensity absorbed and reflected by vascular tissues, allowing it to measure changes in blood volume. These variances are subsequently converted into a waveform, the photoplethysmogram, which forms the basis for downstream analytics. \cite{kyriacou_photoplethysmography_2022}

While photoplethysmogram proves to be a valuable system, it is not devoid of challenges.
In particular, the generated signal is highly susceptible to noise, leading to additive artifacts in the raw signal. Types of noise encountered include power-line interference, external light interference, baseline wandering caused by breathing, and probe-tissue interface disruption \cite{kyriacou_photoplethysmography_2022,elgendi_analysis_2012}. Fortunately, these disturbances are nothing but minor obstacles that can be overcome by well-designed digital filters. 

Digital filtering is an essential measure to reduce the influence of noise. Digital filters refer to time-discrete linear time-invariant (LTI) systems for which there are two main types: Finite Impulse Response (FIR) and Infinite Impulse Response (IIR). As the name suggests, FIR filters only depend on a finite number of input signal, feed-forward samples. In contrast, IIR filters depend additionally on themselves, the feed-back samples, signifying that they can affect the output for an infinite period of time. The response of a filter is determined by its type (low-pass, high-pass, band-pass and band-stop), cutoff frequency and order. Compared to IIR filters, FIR filters have the advantage of stability, numerical robustness and preclusion of signal shape distortions but at the cost of higher orders, requiring increased computational effort and memory space. As a general remark, when filtering in real-time, a delay is always imposed on the signal. \cite{kyriacou_photoplethysmography_2022,noauthor_introduction_nodate}

Given the merits of photoplethysmography as presented earlier and the availability of filters to extract the key features of photoplethysmograms, the aim of this study is to design and implement signal processing algorithms in order to calculate the heart rate in a real-time application using a commercially available PPG front-end and micro-controller development board.










\section{Materials and Methods}


\newcommand{\RNum}[1]{\uppercase\expandafter{\romannumeral #1\relax}}

\section{Results}

\setlength{\tabcolsep}{3pt}
\begin{table}[ht]
	\begin{flushleft}
		\caption{Comparison of segmentation performance between Random Forest and Atlas-based methods.}
		\begin{tabularx}{\linewidth}{Xcccc}
			\toprule
			\multirow{2}{*}{Region} & \multicolumn{2}{c}{Random Forest} & \multicolumn{2}{c}{Atlas} \\ 
			\cmidrule(lr){2-3} \cmidrule(lr){4-5}
			\midrule
			Noise & No & Yes & No & Yes \\
			Amygdala & $0.47 \pm 0.06$ & $0.47 \pm 0.06$ & $\mathbf{0.63 \pm 0.08}$ & $0.61 \pm 0.07$ \\
			Hippocampus & $0.43 \pm 0.05$ & $0.43 \pm 0.06$ & $0.61 \pm 0.10$ & $\mathbf{0.63 \pm 0.06}$ \\
			Thalamus & $0.68 \pm 0.10$ & $0.68 \pm 0.10$ & $\mathbf{0.79\pm 0.04}$ & $0.78 \pm 0.05$ \\
			Grey Matter & $0.73 \pm 0.01$ & $\mathbf{0.74 \pm 0.01}$ & $0.53 \pm 0.02 $ & $0.52 \pm 0.03$ \\
			White Matter & $\mathbf{0.83 \pm 0.02}$ & $\mathbf{0.83 \pm 0.02}$ & $0.66 \pm 0.03$ & $0.66 \pm 0.03$ \\
			Time (s)  & $197.3 \pm 6.3$  & $204.4 \pm 7.3$  & $6.0 \pm 1.8$  & $\mathbf{5.8 \pm 0.7}$ \\
			\bottomrule
		\end{tabularx}
	\end{flushleft}	
	\label{tab:performance_comparison}
\end{table}

The comparison of segmentation performance between the Random Forest-based and Atlas-based methods, both with and without noise, revealed distinct trends across brain regions. These differences can be seen in Table \RNum{1}. For the amygdala, the Atlas-based method without noise achieved the highest Dice score (0.63 ± 0.08) compared to the Random Forest method (0.47 ± 0.06). With noise, the Atlas-based method scored slightly lower (0.61 ± 0.07) but still outperformed the Random Forest method (0.47 ± 0.06). Similarly, for the hippocampus, the Atlas-based method performed better in noisy conditions, achieving the highest Dice score (0.63 ± 0.06) compared to the performance without noise (0.61 ± 0.10), while the Random Forest method showed no variability, scoring consistently (0.43 ± 0.05).
In the thalamus, the Atlas-based method achieved the best performance without noise (0.79 ± 0.04), slightly decreasing in the presence of noise (0.78 ± 0.05), whereas the Random Forest method showed consistent performance (0.68 ± 0.10) across conditions. Contrastingly, for grey matter, the Random Forest method performed better with noise (0.74 ± 0.01) than without noise (0.73 ± 0.01), while the Atlas-based method performed significantly worse, scoring 0.53 ± 0.02 without noise and 0.52 ± 0.03 with noise. For white matter, the Random Forest method outperformed the Atlas-based method with a consistent Dice score of 0.83 ± 0.02 across noise conditions, whereas the Atlas-based method achieved 0.66 ± 0.03 in both cases.
Regarding computational time, the Atlas-based method was significantly faster, taking 6.0 ± 1.8 seconds without noise and 5.8 ± 0.7 seconds with noise. In contrast, the Random Forest method required substantially more time, taking 197.3 ± 6.3 seconds without noise and 209.0 ± 7.5 seconds with noise. These results demonstrate clear differences in segmentation performance and computational efficiency between the two methods.

\section{Discussion}
The results presented in this study provide an insightful comparison between the Random Forest and Atlas-based methods for brain tissue segmentation. These findings reveal that each method has its strengths and weaknesses, which are highly dependent on the target brain structures and the computational resources available.

The Random Forest method demonstrated superior performance in segmenting larger structures, such as white matter and grey matter, primarily due to the type of features used in the classification process. Features capturing intensity, texture, and spatial distribution play a key role in distinguishing tissue types (CITATION NEEDED). Larger structures like white matter tend to exhibit more uniform and distinctive intensity patterns, which are more easily captured by statistical features such as mean intensity and variance. Additionally, textural features, which quantify patterns and variations, are more consistent and pronounced in larger regions, making them easier to classify. In contrast, smaller structures, such as the thalamus, present challenges due to their limited size and less distinct textural properties. The statistical sampling of features in Random Forest models is designed to handle diverse and high-dimensional data, but smaller structures contribute fewer representative samples during the training process. This imbalance can lead to a bias in favor of larger structures that dominate the feature set. Consequently, the Random Forest method's reliance on a multidimensional feature set enables it to excel in segmenting larger regions where spatial and intensity variations are more readily learned, while smaller regions remain more challenging to classify accurately.

The Atlas-based method showed better performance in segmenting smaller structures, such as the thalamus, which exhibit a high degree of consistency in shape, size, and position across subjects (CITATION NEEDED). This anatomical consistency makes these structures particularly well-suited for atlas-based approaches. The compact, ellipsoid shape of the thalamus further simplifies segmentation, reducing the likelihood of errors. Additionally, the method’s reduced sensitivity to minor global misalignments ensures that even if the overall brain alignment is slightly off, the localized region around the thalamus can still align well. When the atlas registration is performed accurately, these factors enable reliable identification of the thalamus and other similar structures (CITATION NEEDED). The inherent simplicity of the atlas-based method, which does not require a complex classifier, enhances its effectiveness for smaller, well-defined regions that are consistently identifiable across subjects. However Atlas-based method had difficulty segmenting grey matter, as evidenced by the lower Dice scores in this region, and also performed poorly with white matter compared to the Random Forest method. Grey matter presents a unique challenge because it is not as homogeneous as other tissue types (CITATION NEEDED), with complex anatomical boundaries and high variability between individuals in terms of size and shape (CITATION NEEDED). Similarly, white matter segmentation suffers from low contrast with neighboring tissues and partial volume effects, which blur boundaries and reduce segmentation accuracy.

Additionally, the atlas-based method's reliance on the quality of the initial atlas can limit its performance for both grey and white matter. If the atlas was constructed from a population with different anatomical characteristics than the study dataset, segmentation accuracy may degrade significantly. The widespread and convoluted nature of grey matter, along with the complex organization of white matter tracts, increases sensitivity to registration errors. Misalignments during registration can lead to compounded inaccuracies, particularly for larger, less consistent structures like white matter. Furthermore, the static nature of the atlas-based method prevents it from adapting dynamically to individual anatomical variability, unlike the Random Forest method, which leverages a large and diverse training dataset to flexibly address such variations and therefore produces more accurate segmentations for both grey and white matter.

In addition to segmentation performance, computational efficiency is a key consideration. The Atlas-based method was significantly faster, completing segmentation in seconds compared to the minutes required by the Random Forest method. This efficiency stems from its reliance on predefined atlases and transformations, involving only resampling and registration steps, making it ideal for large datasets. This speed advantage is critical in clinical settings, where rapid processing is often necessary (CITATION NEEDED). In contrast, the Random Forest method’s extensive training and prediction steps make it more computationally demanding and less suitable for real-time applications without further optimization.


Both methods demonstrated robustness against salt and pepper noise, maintaining segmentation performance despite image disruptions. While it is expected that the Atlas-based method, relying on spatial registration, would handle noise well if the process is robust, the Random Forest method's resilience was more surprising. Notably, the Random Forest performed well on salt and pepper test images even when trained solely on normal data, showcasing its ability to generalize effectively to noisy conditions. This highlights an unexpected strength, particularly valuable in real-world applications where medical images often contain noise from artifacts or patient movement (CITATION NEEDED).

The study has several limitations that must be considered. First, the dataset used in this study may not fully capture the variability in brain anatomy across different populations, such as those with diverse age groups or pathologies. This limitation could affect the generalizability of the findings to broader clinical settings. Second, the Atlas-based method’s performance heavily depends on accurate registration, and misalignments can significantly degrade segmentation quality, particularly in regions with high inter-subject variability. To address this, future work could involve constructing multiple atlases tailored to specific populations, allowing better capture of anatomical variations and improving segmentation accuracy. Lastly, while both methods were robust to salt and pepper noise, their resilience to other noise types, such as Gaussian or motion artifacts, was not evaluated, leaving their performance under such conditions uncertain.
\section{Conclusion}
\section*{Acknowledgment}

%We would like to express our appreciation for the teachings of Professor Mauricio Reyes who, through his engaging and though-provoking lectures for the Medical Image Analysis course, has given us a solid understand of the current knowledge in the field. We would also like to extend our heartfelt thanks to Shelley Shu and Amith Kamath for their guidance throughout this project and for creating a very enjoyable learning environment during the lab.

We would like to extend our heartfelt thanks to Shelley Shu and Amith Kamath for their guidance throughout this project and for fostering an engaging and enjoyable learning atmosphere during the lab. We would also like to express our sincere thanks to Professor Mauricio Reyes for imparting us with valuable knowledge throughout the lectures of the Medical Image Analysis course.
%\section*{References}

\begin{thebibliography}{00}
	\bibitem{b1} “What Is Dementia? Symptoms, Types, and Diagnosis,” National Institute on Aging. Accessed: Nov. 16, 2024. [Online]. Available: https://www.nia.nih.gov/health/alzheimers-and-dementia/what-dementia-symptoms-types-and-diagnosis
	
	\bibitem{b2} “Dementia.” Accessed: Nov. 16, 2024. [Online]. Available: https://www.who.int/news-room/fact-sheets/detail/dementia
	
	\bibitem{b3} A. A. Rostagno, “Pathogenesis of Alzheimer’s Disease,” IJMS, vol. 24, no. 1, p. 107, Dec. 2022, doi: 10.3390/ijms24010107.
	
	\bibitem{b4} K. Gustaw-Rothenberg et al., “Biomarkers in Alzheimer’s Disease: Past, Present and Future,” Biomarkers Med., vol. 4, no. 1, pp. 15–26, Feb. 2010, doi: 10.2217/bmm.09.86.
	
	\bibitem{b5} W. M. Van Oostveen and E. C. M. De Lange, “Imaging Techniques in Alzheimer’s Disease: A Review of Applications in Early Diagnosis and Longitudinal Monitoring,” IJMS, vol. 22, no. 4, p. 2110, Feb. 2021, doi: 10.3390/ijms22042110.
	
	\bibitem{b6} B. Foster, U. Bagci, A. Mansoor, Z. Xu, and D. J. Mollura, “A review on segmentation of positron emission tomography images,” Computers in Biology and Medicine, vol. 50, pp. 76–96, Jul. 2014, doi: 10.1016/j.compbiomed.2014.04.014.
	
	\bibitem{b7} A. W. Toga and P. M. Thompson, “Chapter 43 - Image Registration and the Construction of Multidimensional Brain Atlases,” in Handbook of Medical Image Processing and Analysis (Second Edition), I. N. Bankman, Ed., Academic Press, 2009, pp. 707–724. [Online]. Available: https://www.sciencedirect.com/science/article/pii/B9780123739049500532
	
	\bibitem{b8} C. S. Perone and J. Cohen-Adad, “Promises and limitations of deep learning for medical image segmentation,” J Med Artif Intell, vol. 2, pp. 1–1, Jan. 2019, doi: 10.21037/jmai.2019.01.01.
	
	\bibitem{b9} E. Goceri, “Challenges and Recent Solutions for Image Segmentation in the Era of Deep Learning,” in 2019 Ninth International Conference on Image Processing Theory, Tools and Applications (IPTA), Istanbul, Turkey: IEEE, Nov. 2019, pp. 1–6. doi: 10.1109/IPTA.2019.8936087.
	
	\bibitem{b10} S. Pereira, A. Pinto, J. Oliveira, A. M. Mendrik, J. H. Correia, and C. A. Silva, “Automatic brain tissue segmentation in MR images using Random Forests and Conditional Random Fields,” Journal of Neuroscience Methods, vol. 270, pp. 111–123, Sep. 2016, doi: 10.1016/j.jneumeth.2016.06.017.
	
	\bibitem{b11} “Medical Image Preprocessing.” Accessed: Nov. 30, 2024. [Online]. Available: https://ch.mathworks.com/help/medical-imaging/ug/overview-medical-image-preprocessing.html
	
	\bibitem{b12} “Medical Image Registration.” Accessed: Nov. 30, 2024. [Online]. Available: https://ch.mathworks.com/help/medical-imaging/ug/medical-image-registration.html
	
	\bibitem{b13} M. Salvi, U. R. Acharya, F. Molinari, and K. M. Meiburger, “The impact of pre- and post-image processing techniques on deep learning frameworks: A comprehensive review for digital pathology image analysis,” Computers in Biology and Medicine, vol. 128, p. 104129, Jan. 2021, doi: 10.1016/j.compbiomed.2020.104129.
	
	\bibitem{b14} “Understanding Evaluation Metrics in Medical Image Segmentation,” Medium. Accessed: Nov. 30, 2024. [Online]. Available: https://medium.com/@nghihuynh\_37300/understanding-evaluation-metrics-in-medical-image-segmentation-d289a373a3f
	

\end{thebibliography}







% Start of annex/appendices
%\clearpage
%\newpage
%\onecolumn % Switch to single-column mode
%\appendix
%\section{Appendix}
%\begin{figure} [H]
	\centering
	\includegraphics[width=\textwidth]{"figures/atlas before median filter"}
	\caption{Atlas Labels before Morphological Filters.}
	\label{fig:atlas-before-median-filter}
\end{figure}

\begin{figure} [H]
	\centering
	\includegraphics[width=\textwidth]{"figures/atlas after filter"}
	\caption{Atlas Labels after Morphological Filters.}
	\label{fig:atlas-after-median-filter}
\end{figure}

\begin{figure} [H]
	\centering
	\includegraphics[width=\textwidth]{"figures/atlas_best"}
	\caption{Best Overall Performing Atlas Segmentation Labels, Patient 123117.}
	\label{fig:atlas-best}
\end{figure}

\begin{figure} [H]
	\centering
	\includegraphics[width=\textwidth]{"figures/atlas_worst"}
	\caption{Worst Overall Performing Atlas-based Segmentation Labels, Patient 124422.}
	\label{fig:atlas-worst}
\end{figure}

\begin{figure} [H]
	\centering
	\includegraphics[width=\textwidth]{"figures/rf_best"}
	\caption{Best Overall Performing Random Forest Segmentation Labels, Patient 124422.}
	\label{fig:rf-best}
\end{figure}

\begin{figure} [H]
	\centering
	\includegraphics[width=\textwidth]{"figures/rf_worst"}
	\caption{Worst Overall Performing Random Forest Segmentation Labels, Patient 123117.}
	\label{fig:rf-worst}
\end{figure}




\end{document}
